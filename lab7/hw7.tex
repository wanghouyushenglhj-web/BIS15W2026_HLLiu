% Options for packages loaded elsewhere
\PassOptionsToPackage{unicode}{hyperref}
\PassOptionsToPackage{hyphens}{url}
\documentclass[
]{article}
\usepackage{xcolor}
\usepackage[margin=1in]{geometry}
\usepackage{amsmath,amssymb}
\setcounter{secnumdepth}{-\maxdimen} % remove section numbering
\usepackage{iftex}
\ifPDFTeX
  \usepackage[T1]{fontenc}
  \usepackage[utf8]{inputenc}
  \usepackage{textcomp} % provide euro and other symbols
\else % if luatex or xetex
  \usepackage{unicode-math} % this also loads fontspec
  \defaultfontfeatures{Scale=MatchLowercase}
  \defaultfontfeatures[\rmfamily]{Ligatures=TeX,Scale=1}
\fi
\usepackage{lmodern}
\ifPDFTeX\else
  % xetex/luatex font selection
\fi
% Use upquote if available, for straight quotes in verbatim environments
\IfFileExists{upquote.sty}{\usepackage{upquote}}{}
\IfFileExists{microtype.sty}{% use microtype if available
  \usepackage[]{microtype}
  \UseMicrotypeSet[protrusion]{basicmath} % disable protrusion for tt fonts
}{}
\makeatletter
\@ifundefined{KOMAClassName}{% if non-KOMA class
  \IfFileExists{parskip.sty}{%
    \usepackage{parskip}
  }{% else
    \setlength{\parindent}{0pt}
    \setlength{\parskip}{6pt plus 2pt minus 1pt}}
}{% if KOMA class
  \KOMAoptions{parskip=half}}
\makeatother
\usepackage{color}
\usepackage{fancyvrb}
\newcommand{\VerbBar}{|}
\newcommand{\VERB}{\Verb[commandchars=\\\{\}]}
\DefineVerbatimEnvironment{Highlighting}{Verbatim}{commandchars=\\\{\}}
% Add ',fontsize=\small' for more characters per line
\usepackage{framed}
\definecolor{shadecolor}{RGB}{248,248,248}
\newenvironment{Shaded}{\begin{snugshade}}{\end{snugshade}}
\newcommand{\AlertTok}[1]{\textcolor[rgb]{0.94,0.16,0.16}{#1}}
\newcommand{\AnnotationTok}[1]{\textcolor[rgb]{0.56,0.35,0.01}{\textbf{\textit{#1}}}}
\newcommand{\AttributeTok}[1]{\textcolor[rgb]{0.13,0.29,0.53}{#1}}
\newcommand{\BaseNTok}[1]{\textcolor[rgb]{0.00,0.00,0.81}{#1}}
\newcommand{\BuiltInTok}[1]{#1}
\newcommand{\CharTok}[1]{\textcolor[rgb]{0.31,0.60,0.02}{#1}}
\newcommand{\CommentTok}[1]{\textcolor[rgb]{0.56,0.35,0.01}{\textit{#1}}}
\newcommand{\CommentVarTok}[1]{\textcolor[rgb]{0.56,0.35,0.01}{\textbf{\textit{#1}}}}
\newcommand{\ConstantTok}[1]{\textcolor[rgb]{0.56,0.35,0.01}{#1}}
\newcommand{\ControlFlowTok}[1]{\textcolor[rgb]{0.13,0.29,0.53}{\textbf{#1}}}
\newcommand{\DataTypeTok}[1]{\textcolor[rgb]{0.13,0.29,0.53}{#1}}
\newcommand{\DecValTok}[1]{\textcolor[rgb]{0.00,0.00,0.81}{#1}}
\newcommand{\DocumentationTok}[1]{\textcolor[rgb]{0.56,0.35,0.01}{\textbf{\textit{#1}}}}
\newcommand{\ErrorTok}[1]{\textcolor[rgb]{0.64,0.00,0.00}{\textbf{#1}}}
\newcommand{\ExtensionTok}[1]{#1}
\newcommand{\FloatTok}[1]{\textcolor[rgb]{0.00,0.00,0.81}{#1}}
\newcommand{\FunctionTok}[1]{\textcolor[rgb]{0.13,0.29,0.53}{\textbf{#1}}}
\newcommand{\ImportTok}[1]{#1}
\newcommand{\InformationTok}[1]{\textcolor[rgb]{0.56,0.35,0.01}{\textbf{\textit{#1}}}}
\newcommand{\KeywordTok}[1]{\textcolor[rgb]{0.13,0.29,0.53}{\textbf{#1}}}
\newcommand{\NormalTok}[1]{#1}
\newcommand{\OperatorTok}[1]{\textcolor[rgb]{0.81,0.36,0.00}{\textbf{#1}}}
\newcommand{\OtherTok}[1]{\textcolor[rgb]{0.56,0.35,0.01}{#1}}
\newcommand{\PreprocessorTok}[1]{\textcolor[rgb]{0.56,0.35,0.01}{\textit{#1}}}
\newcommand{\RegionMarkerTok}[1]{#1}
\newcommand{\SpecialCharTok}[1]{\textcolor[rgb]{0.81,0.36,0.00}{\textbf{#1}}}
\newcommand{\SpecialStringTok}[1]{\textcolor[rgb]{0.31,0.60,0.02}{#1}}
\newcommand{\StringTok}[1]{\textcolor[rgb]{0.31,0.60,0.02}{#1}}
\newcommand{\VariableTok}[1]{\textcolor[rgb]{0.00,0.00,0.00}{#1}}
\newcommand{\VerbatimStringTok}[1]{\textcolor[rgb]{0.31,0.60,0.02}{#1}}
\newcommand{\WarningTok}[1]{\textcolor[rgb]{0.56,0.35,0.01}{\textbf{\textit{#1}}}}
\usepackage{graphicx}
\makeatletter
\newsavebox\pandoc@box
\newcommand*\pandocbounded[1]{% scales image to fit in text height/width
  \sbox\pandoc@box{#1}%
  \Gscale@div\@tempa{\textheight}{\dimexpr\ht\pandoc@box+\dp\pandoc@box\relax}%
  \Gscale@div\@tempb{\linewidth}{\wd\pandoc@box}%
  \ifdim\@tempb\p@<\@tempa\p@\let\@tempa\@tempb\fi% select the smaller of both
  \ifdim\@tempa\p@<\p@\scalebox{\@tempa}{\usebox\pandoc@box}%
  \else\usebox{\pandoc@box}%
  \fi%
}
% Set default figure placement to htbp
\def\fps@figure{htbp}
\makeatother
\setlength{\emergencystretch}{3em} % prevent overfull lines
\providecommand{\tightlist}{%
  \setlength{\itemsep}{0pt}\setlength{\parskip}{0pt}}
\usepackage{bookmark}
\IfFileExists{xurl.sty}{\usepackage{xurl}}{} % add URL line breaks if available
\urlstyle{same}
\hypersetup{
  pdftitle={Homework 7},
  pdfauthor={Your Name Here},
  hidelinks,
  pdfcreator={LaTeX via pandoc}}

\title{Homework 7}
\author{Your Name Here}
\date{2026-01-27}

\begin{document}
\maketitle

\subsection{Instructions}\label{instructions}

Answer the following questions and/or complete the exercises in
RMarkdown. Please embed all of your code and push the final work to your
repository. Your report should be organized, clean, and run free from
errors. Remember, you must remove the \texttt{\#} for any included code
chunks to run.

\subsection{Load the tidyverse}\label{load-the-tidyverse}

\begin{Shaded}
\begin{Highlighting}[]
\FunctionTok{library}\NormalTok{(}\StringTok{"tidyverse"}\NormalTok{)}
\FunctionTok{library}\NormalTok{(}\StringTok{"janitor"}\NormalTok{)}
\end{Highlighting}
\end{Shaded}

\subsection{Data}\label{data}

For this assignment, we will use data from a study on elephants and the
effects of poaching on tusk size.

Reference: \href{https://doi.org/10.1002/ece3.1769}{Chiyo, Patrick I.,
Vincent Obanda, and David K. Korir. ``Illegal tusk harvest and the
decline of tusk size in the African elephant.'' Ecology and Evolution 5,
22: 5216--5229 (2015)}. Data deposited at
\href{https://doi.org/10.5061/dryad.h6t7j}{Dryad Digital Repository}.

\textbf{1. Before starting data analysis, read the abstract of the paper
to get an idea of the questions being asked. In 2-3 sentences, describe
what the study is testing and the variables involved.}\\
\strut \\
\This study is trying to find out if the illegal tusk harvest have a
signigicant effect on the tusk size over the natural selection.variables
involved:tusk size,sex,shoulder height.\\
\strut \\
\strut \\

\textbf{2. Load \texttt{elephants.csv} and store it as a new object
called \texttt{elephants}.}\\
\strut \\
\strut \\

\begin{Shaded}
\begin{Highlighting}[]
\NormalTok{elephant }\OtherTok{\textless{}{-}} \FunctionTok{read\_csv}\NormalTok{(}\StringTok{"data/elephants.csv"}\NormalTok{)}
\end{Highlighting}
\end{Shaded}

\begin{verbatim}
## Rows: 777 Columns: 7
## -- Column specification --------------------------------------------------------
## Delimiter: ","
## chr (3): Years of sample collection, Elephant ID, Sex
## dbl (4): Estimated Age (years), shoulder Height in  cm, Tusk Length in cm, T...
## 
## i Use `spec()` to retrieve the full column specification for this data.
## i Specify the column types or set `show_col_types = FALSE` to quiet this message.
\end{verbatim}

\hfill\break
\hfill\break
\hfill\break

\textbf{3. Clean the data by converting variable names to lowercase with
no spaces or special characters.}\\
\strut \\
\strut \\

\begin{Shaded}
\begin{Highlighting}[]
\NormalTok{elephant }\OtherTok{\textless{}{-}}\NormalTok{ elephant}\SpecialCharTok{\%\textgreater{}\%} 
  \FunctionTok{clean\_names}\NormalTok{() }\SpecialCharTok{\%\textgreater{}\%} 
  \FunctionTok{mutate}\NormalTok{(}\FunctionTok{across}\NormalTok{(}\FunctionTok{where}\NormalTok{(is.character),tolower))}
\end{Highlighting}
\end{Shaded}

\hfill\break
\hfill\break

\textbf{4. Use one or more of the summary functions you have learned to
get an idea of the structure of the data.}

\begin{Shaded}
\begin{Highlighting}[]
\FunctionTok{summary}\NormalTok{(elephant)}
\end{Highlighting}
\end{Shaded}

\begin{verbatim}
##  years_of_sample_collection elephant_id            sex           
##  Length:777                 Length:777         Length:777        
##  Class :character           Class :character   Class :character  
##  Mode  :character           Mode  :character   Mode  :character  
##                                                                  
##                                                                  
##                                                                  
##                                                                  
##  estimated_age_years shoulder_height_in_cm tusk_length_in_cm
##  Min.   : 0.08       Min.   : 89.0         Min.   : 22.50   
##  1st Qu.: 4.50       1st Qu.:177.0         1st Qu.: 60.95   
##  Median :12.00       Median :220.0         Median : 88.00   
##  Mean   :15.05       Mean   :210.2         Mean   : 91.59   
##  3rd Qu.:23.00       3rd Qu.:244.0         3rd Qu.:116.75   
##  Max.   :55.00       Max.   :340.0         Max.   :234.00   
##  NA's   :3           NA's   :1             NA's   :180      
##  tusk_circumference_in_cm
##  Min.   : 8.00           
##  1st Qu.:16.00           
##  Median :20.00           
##  Mean   :21.01           
##  3rd Qu.:25.00           
##  Max.   :48.00           
##  NA's   :163
\end{verbatim}

\begin{Shaded}
\begin{Highlighting}[]
\FunctionTok{str}\NormalTok{(elephant)}
\end{Highlighting}
\end{Shaded}

\begin{verbatim}
## tibble [777 x 7] (S3: tbl_df/tbl/data.frame)
##  $ years_of_sample_collection: chr [1:777] "1966-68" "1966-68" "1966-68" "1966-68" ...
##  $ elephant_id               : chr [1:777] "12" "34" "162" "292" ...
##  $ sex                       : chr [1:777] "f" "f" "f" "f" ...
##  $ estimated_age_years       : num [1:777] 0.08 0.08 0.083 0.083 0.25 0.25 0.25 0.5 0.5 1 ...
##  $ shoulder_height_in_cm     : num [1:777] 102 89 89 92 133 100 93 108 108 124 ...
##  $ tusk_length_in_cm         : num [1:777] NA NA NA NA NA NA NA NA NA NA ...
##  $ tusk_circumference_in_cm  : num [1:777] NA NA NA NA NA NA NA NA NA NA ...
\end{verbatim}

\begin{Shaded}
\begin{Highlighting}[]
\FunctionTok{glimpse}\NormalTok{(elephant)}
\end{Highlighting}
\end{Shaded}

\begin{verbatim}
## Rows: 777
## Columns: 7
## $ years_of_sample_collection <chr> "1966-68", "1966-68", "1966-68", "1966-68",~
## $ elephant_id                <chr> "12", "34", "162", "292", "11", "152", "264~
## $ sex                        <chr> "f", "f", "f", "f", "f", "f", "f", "f", "f"~
## $ estimated_age_years        <dbl> 0.080, 0.080, 0.083, 0.083, 0.250, 0.250, 0~
## $ shoulder_height_in_cm      <dbl> 102, 89, 89, 92, 133, 100, 93, 108, 108, 12~
## $ tusk_length_in_cm          <dbl> NA, NA, NA, NA, NA, NA, NA, NA, NA, NA, NA,~
## $ tusk_circumference_in_cm   <dbl> NA, NA, NA, NA, NA, NA, NA, NA, NA, NA, NA,~
\end{verbatim}

\textbf{5. Use \texttt{mutate()} Change the variables
\texttt{years\_of\_sample\_collection}, \texttt{elephant\_id}, and
\texttt{sex} to factors. Be sure to store the output as a new dataframe
and use it for the remaining questions.}\\
\strut \\

\begin{Shaded}
\begin{Highlighting}[]
\NormalTok{elephant }\SpecialCharTok{\%\textgreater{}\%} 
  \FunctionTok{mutate}\NormalTok{(}\FunctionTok{across}\NormalTok{(}\FunctionTok{c}\NormalTok{(years\_of\_sample\_collection,elephant\_id,sex),as.factor))}
\end{Highlighting}
\end{Shaded}

\begin{verbatim}
## # A tibble: 777 x 7
##    years_of_sample_collection elephant_id sex   estimated_age_years
##    <fct>                      <fct>       <fct>               <dbl>
##  1 1966-68                    12          f                   0.08 
##  2 1966-68                    34          f                   0.08 
##  3 1966-68                    162         f                   0.083
##  4 1966-68                    292         f                   0.083
##  5 1966-68                    11          f                   0.25 
##  6 1966-68                    152         f                   0.25 
##  7 1966-68                    264         f                   0.25 
##  8 1966-68                    263         f                   0.5  
##  9 1966-68                    266         f                   0.5  
## 10 1966-68                    217         f                   1    
## # i 767 more rows
## # i 3 more variables: shoulder_height_in_cm <dbl>, tusk_length_in_cm <dbl>,
## #   tusk_circumference_in_cm <dbl>
\end{verbatim}

\hfill\break
\hfill\break
\hfill\break

\textbf{6. From which years were data collected? Show the sample periods
below.}\\
\strut \\
\strut \\

\begin{Shaded}
\begin{Highlighting}[]
\NormalTok{elephant }\SpecialCharTok{\%\textgreater{}\%} 
  \FunctionTok{count}\NormalTok{(years\_of\_sample\_collection)}
\end{Highlighting}
\end{Shaded}

\begin{verbatim}
## # A tibble: 2 x 2
##   years_of_sample_collection     n
##   <chr>                      <int>
## 1 1966-68                      605
## 2 2005-13                      172
\end{verbatim}

\hfill\break
\hfill\break

\textbf{7. How many males and females were sampled in this study?}\\
\strut \\
\strut \\
\strut \\
\strut \\
\strut \\

\begin{Shaded}
\begin{Highlighting}[]
\NormalTok{elephant }\SpecialCharTok{\%\textgreater{}\%} 
  \FunctionTok{count}\NormalTok{(sex)}
\end{Highlighting}
\end{Shaded}

\begin{verbatim}
## # A tibble: 2 x 2
##   sex       n
##   <chr> <int>
## 1 f       416
## 2 m       361
\end{verbatim}

\textbf{8. What is the mean, median, and standard deviation for age of
males and females included in the study? Separate the results by year of
sample collection. Does the sampling look even between years and sexes?}

\begin{Shaded}
\begin{Highlighting}[]
\FunctionTok{names}\NormalTok{(elephant)}
\end{Highlighting}
\end{Shaded}

\begin{verbatim}
## [1] "years_of_sample_collection" "elephant_id"               
## [3] "sex"                        "estimated_age_years"       
## [5] "shoulder_height_in_cm"      "tusk_length_in_cm"         
## [7] "tusk_circumference_in_cm"
\end{verbatim}

\#meand,median,sd for age of males

\begin{Shaded}
\begin{Highlighting}[]
\NormalTok{elephant }\SpecialCharTok{\%\textgreater{}\%} 
  \FunctionTok{filter}\NormalTok{(sex}\SpecialCharTok{==}\StringTok{"f"}\NormalTok{,years\_of\_sample\_collection}\SpecialCharTok{==}\StringTok{"2005{-}13"}\NormalTok{) }\SpecialCharTok{\%\textgreater{}\%} 
  \FunctionTok{summarise}\NormalTok{(}\AttributeTok{mean\_age\_f=}\FunctionTok{mean}\NormalTok{(estimated\_age\_years,}\AttributeTok{na.rm=}\NormalTok{T),}
            \AttributeTok{median\_age\_f=}\FunctionTok{median}\NormalTok{(estimated\_age\_years,}\AttributeTok{na.rm=}\NormalTok{T),}
            \AttributeTok{sd\_age\_f=}\FunctionTok{sd}\NormalTok{(estimated\_age\_years,}\AttributeTok{na.rm=}\NormalTok{T))}\CommentTok{\#meand,median,sd for age of females during year of 2005{-}13}
\end{Highlighting}
\end{Shaded}

\begin{verbatim}
## # A tibble: 1 x 3
##   mean_age_f median_age_f sd_age_f
##        <dbl>        <dbl>    <dbl>
## 1       17.9         17.5     11.0
\end{verbatim}

\hfill\break
\#meand,median,sd for age of females

\begin{Shaded}
\begin{Highlighting}[]
\NormalTok{elephant }\SpecialCharTok{\%\textgreater{}\%} 
  \FunctionTok{filter}\NormalTok{(sex}\SpecialCharTok{==}\StringTok{"f"}\NormalTok{,years\_of\_sample\_collection}\SpecialCharTok{==}\StringTok{"1966{-}68"}\NormalTok{) }\SpecialCharTok{\%\textgreater{}\%} 
  \FunctionTok{summarise}\NormalTok{(}\AttributeTok{mean\_age\_f=}\FunctionTok{mean}\NormalTok{(estimated\_age\_years,}\AttributeTok{na.rm=}\NormalTok{T),}
            \AttributeTok{median\_age\_f=}\FunctionTok{median}\NormalTok{(estimated\_age\_years,}\AttributeTok{na.rm=}\NormalTok{T),}
            \AttributeTok{sd\_age\_f=}\FunctionTok{sd}\NormalTok{(estimated\_age\_years,}\AttributeTok{na.rm=}\NormalTok{T))}\CommentTok{\#meand,median,sd for age of females during year of 1966{-}68}
\end{Highlighting}
\end{Shaded}

\begin{verbatim}
## # A tibble: 1 x 3
##   mean_age_f median_age_f sd_age_f
##        <dbl>        <dbl>    <dbl>
## 1       17.6           15     13.6
\end{verbatim}

\begin{Shaded}
\begin{Highlighting}[]
\NormalTok{elephant }\SpecialCharTok{\%\textgreater{}\%} 
  \FunctionTok{filter}\NormalTok{(sex}\SpecialCharTok{==}\StringTok{"m"}\NormalTok{,years\_of\_sample\_collection}\SpecialCharTok{==}\StringTok{"1966{-}68"}\NormalTok{) }\SpecialCharTok{\%\textgreater{}\%} 
  \FunctionTok{summarise}\NormalTok{(}\AttributeTok{mean\_age\_m=}\FunctionTok{mean}\NormalTok{(estimated\_age\_years,}\AttributeTok{na.rm=}\NormalTok{T),}
            \AttributeTok{median\_age\_m=}\FunctionTok{median}\NormalTok{(estimated\_age\_years,}\AttributeTok{na.rm=}\NormalTok{T),}
            \AttributeTok{sd\_age\_m=}\FunctionTok{sd}\NormalTok{(estimated\_age\_years,}\AttributeTok{na.rm=}\NormalTok{T))}\CommentTok{\#meand,median,sd for age of males during year of 1966{-}68}
\end{Highlighting}
\end{Shaded}

\begin{verbatim}
## # A tibble: 1 x 3
##   mean_age_m median_age_m sd_age_m
##        <dbl>        <dbl>    <dbl>
## 1       10.8            8     9.19
\end{verbatim}

\begin{Shaded}
\begin{Highlighting}[]
\NormalTok{elephant }\SpecialCharTok{\%\textgreater{}\%} 
  \FunctionTok{filter}\NormalTok{(sex}\SpecialCharTok{==}\StringTok{"m"}\NormalTok{,years\_of\_sample\_collection}\SpecialCharTok{==}\StringTok{"2005{-}13"}\NormalTok{) }\SpecialCharTok{\%\textgreater{}\%} 
  \FunctionTok{summarise}\NormalTok{(}\AttributeTok{mean\_age\_m=}\FunctionTok{mean}\NormalTok{(estimated\_age\_years,}\AttributeTok{na.rm=}\NormalTok{T),}
            \AttributeTok{median\_age\_m=}\FunctionTok{median}\NormalTok{(estimated\_age\_years,}\AttributeTok{na.rm=}\NormalTok{T),}
            \AttributeTok{sd\_age\_m=}\FunctionTok{sd}\NormalTok{(estimated\_age\_years,}\AttributeTok{na.rm=}\NormalTok{T))}\CommentTok{\#meand,median,sd for age of males during year of 2005{-}13}
\end{Highlighting}
\end{Shaded}

\begin{verbatim}
## # A tibble: 1 x 3
##   mean_age_m median_age_m sd_age_m
##        <dbl>        <dbl>    <dbl>
## 1       16.7            9     13.9
\end{verbatim}

\textbf{9. Is age (independent variable) a positive predictor of tusk
length (dependent variable)? Create a plot that shows the relationship
between these variables and add a linear model fit line.}\\
\strut \\
\strut \\

\begin{Shaded}
\begin{Highlighting}[]
\NormalTok{elephant }\SpecialCharTok{\%\textgreater{}\%} 
  \FunctionTok{ggplot}\NormalTok{(}\AttributeTok{mapping=}\FunctionTok{aes}\NormalTok{(}\AttributeTok{x=}\NormalTok{estimated\_age\_years,}\AttributeTok{y=}\NormalTok{tusk\_length\_in\_cm))}\SpecialCharTok{+}
  \FunctionTok{geom\_point}\NormalTok{(}\AttributeTok{mapping =}\FunctionTok{aes}\NormalTok{(}\AttributeTok{color=}\NormalTok{tusk\_length\_in\_cm))}\SpecialCharTok{+}
  \FunctionTok{geom\_smooth}\NormalTok{(}\AttributeTok{method =}\NormalTok{ lm)}\SpecialCharTok{+}
  \FunctionTok{labs}\NormalTok{(}\AttributeTok{title=}\StringTok{"Ages vs tusk\_length"}\NormalTok{)}
\end{Highlighting}
\end{Shaded}

\begin{verbatim}
## `geom_smooth()` using formula = 'y ~ x'
\end{verbatim}

\begin{verbatim}
## Warning: Removed 182 rows containing non-finite outside the scale range
## (`stat_smooth()`).
\end{verbatim}

\begin{verbatim}
## Warning: Removed 182 rows containing missing values or values outside the scale range
## (`geom_point()`).
\end{verbatim}

\pandocbounded{\includegraphics[keepaspectratio]{hw7_files/figure-latex/unnamed-chunk-16-1.pdf}}

\hfill\break
\hfill\break

\textbf{10. Is shoulder height (independent variable) a positive
predictor of tusk length (dependent variable)? Create a plot that shows
the relationship between these variables and add a linear model fit
line.}\\
\strut \\
\strut \\
\strut \\
\strut \\
\strut \\

\textbf{11. The authors argue that because poachers preferentially
target elephants with large tusks, this has resulted in a decrease in
average tusk length. Is this supported by the data? Show your code and
calculations below.}\\
\strut \\
\strut \\
\strut \\
\strut \\
\strut \\

\textbf{12. Male elephants reach effective sexual maturity at 25 years
while females are sexually mature at 12 years. Make a new dataframe that
extracts only the males and females at sexual maturity. Then, make a
plot that shows the range of tusk length between the two sample periods
for these mature elephants.}\\
\strut \\
\strut \\
\strut \\
\strut \\
\strut \\

\subsection{Submit the Homework}\label{submit-the-homework}

\begin{enumerate}
\def\labelenumi{\arabic{enumi}.}
\tightlist
\item
  Save your work and knit the .rmd file.\\
\item
  Open the .html file and ``print'' it to a .pdf file in Google Chrome
  (not Safari).\\
\item
  Go to the class Canvas page and open Gradescope.\\
\item
  Submit your .pdf file to the homework assignment- be sure to assign
  the pages to the correct questions.\\
\item
  Commit and push your work to your repository.
\end{enumerate}

\end{document}
